\documentclass[10pt,a4paper]{article}
%\documentclass{book}
\usepackage[latin1]{inputenc}
\usepackage[spanish]{babel}
\usepackage{amsmath}
\usepackage{amsfonts}
\usepackage{amssymb}
\usepackage{makeidx}
\usepackage{graphicx}
\usepackage{lmodern}
\usepackage[left=3cm,right=3cm,top=4cm,bottom=4cm]{geometry}

%----------------------------------------------------------------------------------------------
% TITLE PAGE
%----------------------------------------------------------------------------------------------

\newcommand{\titleGP}{\begingroup % Create the command for including the title page in the document
\centering % Center all text
\vspace*{\baselineskip} % White space at the top of the page

\rule{\textwidth}{1.6pt}\vspace*{-\baselineskip}\vspace*{2pt} %Thick horizontal line
\rule{\textwidth}{0.4pt}\\[\baselineskip] % Thin horizontal line

{\LARGE An\'alisis Exploratorio}\\[0.2\baselineskip] % Title

\rule{\textwidth}{0.4pt}\vspace*{-\baselineskip}\vspace{3.2pt} %Thin horizontal line
\rule{\textwidth}{1.6pt}\\[\baselineskip] %Thick horizontal line

\scshape %Small caps
%Se puede escribir algo o no

\vspace*{10\baselineskip}

Autor \\[\baselineskip]
{\LARGE Garagiola Mariano\par}


\vspace*{1.5\baselineskip}
{\itshape C\'ordoba \par}

\vfill

\endgroup}

\begin{document}
 \pagestyle{empty}
 \titleGP
 
 \newpage
 
 En este informe presento un breve an\'alisis explorario del {\em dataframe:} {\em historico-nombre.csv}.
 
El dataframe contiene los nombres registrados en el Registro Nacional de 
las Personas agrupados por a\~no, desde el a\~no 1922 hasta el a\~no 2015. Quiero 
mencionar que debido a la cantidad total de nombres registrados no pude realizar un 
an\'alisis del conjunto completo debido a las limitaciones computacionales. El 
an\'alisis que presento lo realic\'e con los primeros cinco millones de datos 
(5432786 datos exactamente), de esta 
forma, cubro el periodo temporal comprendido entre 1922 y 1988.

Inici\'e contando la cantidad de personas registradas por a\~no en 
el Registro Nacional de las Personas. El gr\'afico est\'a presentado en la figura 
\ref{fig1}, donde se puede observar un crecimiento en el n\'umero de personas registradas 
a medida que se avanza en el tiempo, mostrando un claro crecimiento en la poblaci\'on.

\begin{figure}[h!]
\begin{center}
\includegraphics[scale=0.3]{../num_personas_por_anio.pdf}
\end{center}
\caption{\label{fig1} N\'umero de personas registradas por a\~no en el Registro 
Nacional de las Pesonas.}
\end{figure}

\noindent Si bien, en general, se muestra un crecimiento en la cantidad de personas 
registradas, hay algunos a\~nos en los cuales dicho n\'umero disminuye respecto al 
a\~no anterior o bien no sigue con la tendencia, esto probablemente es debido a 
condiciones pol\'iticas-sociales del pa\'is durante esos a\~nos.

A continuaci\'on, decid\'i analizar la cantidad de nombres distintos registrados a lo 
largo de los a\~nos. Dicho gr\'afico, figura \ref{fig2}, muestra que tambi\'en ha 
aumentado la variedad de nombres a lo largo del tiempo, esta cantidad tiene un 
comportamiento similar al mostrando en la figura \ref{fig1}. Nuevamente, hay a\~nos en 
los cuales la cantidad de nombres distintos registrados no sigue la tendencia de los 
a\~nos anteriores, como por ejemplo el a\~no 1980, pero \'esto no quiere decir que 
haya disminuido la cantidad de personas registradas. Ser\'ia interesante poder tener 
los registros de la cantidad de personas que nacieron en cada a\~no, de esa forma 
ser\'ia posible estimar el porcentaje de personas nacidas en Argentina y as\'i tambi\'en 
estimar el porcentaje de personas resgistradas que provienen de otros pa\'ises, ya que 
hubo a\~nos en los cuales inmigraron un gran cantidad de personas.

\begin{figure}[t]
\begin{center}
\includegraphics[scale=0.3]{../num_nombres_por_anio.pdf}
\end{center}
\caption{\label{fig2} N\'umero de nombres distintos registradas por a\~no en el Registro 
Nacional de las Pesonas.}
\end{figure}

Un hecho que me llam\'o la atenci\'on del {\em dataset} es que algunos nombres est\'an 
formados por muchas palabras, por este motivo decid\'i determinar el n\'umero de 
palabras que forma cada nombre. De esta forma, realic\'e el gr\'afico del n\'umero de 
personas con una dada cantidad de palabras en su respectivo nombre, el gr\'afico se 
puede ver en la figura \ref{fig3}. Como es de esperarse, la mayor cantidad de nombres est\'a 
compuesto por dos palabras, pero hay un n\'umero considerable de nombres con una, tres y cuatro 
palabras, lo cual es entendible. En el registro hay personas con m\'as de cinco palabras en el nombre, 
si bien
no est\'a prohibido tener m\'as de cinco palabras en el nombre no deja de ser extra\~no. En particular, 
existen cuatro registros que contienen diez palabras en el nombre y dos registros que contienen once 
palabras en el nombre. Viendo en detalle cada uno, encontr\'e que uno de los registros tiene el nombre:
{\em Maria De La Concepcion Trinidad Maria Amalia Fernanda Felisa Ignacia Adelaida} y es del a\~no 1922, 
lo que me hace dudar de que sea una misma persona. Lo mismo sucede con el segundo registro que posee 
once palabras, el nombre es {\em Josina Livina Dolores Lucia Carmen Balduina Yolanda Maria Inmaculada 
Gislena Javiera}.

\begin{figure}[h!]
\begin{center}
\includegraphics[scale=0.3]{../num_nombres.pdf}
\end{center}
\caption{\label{fig3} N\'umero de personas con un dado n\'umero de palabras en su 
nombre.}
\end{figure}

A partir del {\em dataset} es posible determinar cu\'al fue el nombre m\'as usado en cada a\~no, por 
ejemplo, Juan Carlos fue el nombre m\'as usado durante los a\~nos de 1933 al 1951. 

\end{document}
